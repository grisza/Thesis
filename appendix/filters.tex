\chapter{Popularne metody filtracji i estymacji sygnałów pomiarowych}
\label{chap:appx:filters}
Istotnym etapem przetwarzania sygnałów jest ich filtracja. Zazwyczaj celem tej operacji jest usunięcie niepożądanego szumu z~uzyskanych pomiarów.~Znanych jest wiele metod filtracji, które weszły do powszechnego użytku, jednak szczególną grupą filtrów są filtry Kalmana.~Przez ponad 50 lat znalazły one zastosowanie w~przemyśle, a~także stały się obiektem wielu\footnote{W samej tylko bibliotece cyfrowej IEEE Xplore (\url{http://ieeexplore.ieee.org}) wyszukiwanie frazy ''Kalman filter'' zwraca około 25000 wyników.} badań i~opracowań naukowych \cite{Droleta,Welch2006, Kedzierski2007, Huo2014, Pandey2014}.

\section*{Liniowy filtr Kalmana (KF - \emph{ang. Kalman filter})}
\label{sec:appx:filters:KF}
\nomenclature{KF}{Liniowy filtr Kalmana}
Liniowy filtr Kalmana jest oryginalną metodą filtracji sygnału opracowana i~opisaną przez Rudolpha E. Kalmana w~1960 \cite{Kalman1960} i~1961 \cite{KalmanBucy1961}. Metoda ta pozwala na oszacowanie wartości kolejnych pomiarów tylko na podstawie bieżących danych oraz modelu matematycznego obserwowanego zjawiska. Działanie filtra możemy podzielić na dwie następujące bezpośrednio po sobie fazy: 
\begin{enumerate}
	\item Predykacji,
	\item Korekcji.
\end{enumerate}

Faza pierwsza polega na oszacowaniu aktualnej wartości (stanu) pomiaru na podstawie ostatniego znanego pomiaru oraz modelu przejścia pomiędzy kolejnymi stanami. Jeśli w~naszym układzie występuje źródło sygnału kontrolnego, pomiary z~niego uzyskane możemy wykorzystać w~tej fazie do wymuszenia określonego stanu. Oszacowanie aktualnego pomiaru uwzględnia także występowanie szumu, który to z~założenia jest szumem białym. Równania \ref{eq:appx:KF:ph1a} oraz \ref{eq:appx:KF:ph1b} przedstawiają matematyczny zapis fazy predykacji nazywanej także aktualizacją czasową.

\begin{subequations}
	\begin{align}
		\widehat{x}^-_k & = A\widehat{x}_{k-1} + Bu_{k-1} + w_{k-1}\label{eq:appx:KF:ph1a} \\
		P^-_k           & = AP_{k-1}A^T + Q \label{eq:appx:KF:ph1b}                        
	\end{align}
\end{subequations}

gdzie:
\begin{itemize}
	\item $\widehat{x}^-_k$ -- prognozowana wartość stanu w kroku $k$,
	\item $\widehat{x}_{k-1}$ -- wartość stanu oszacowana w~kroku $k-1$,
	\item $P^-_k$ -- prognozowana wartość kowariancji estymacji,
	\item $P_{k-1}$ -- faktyczna wartość kowariancji estymacji wykonana w~kroku $k-1$,
	\item $A$ -- stała macierz definiująca przejście pomiędzy dwoma, następującymi po sobie stanami,
	\item $B$ --  stała macierz definiująca powiązanie pomiędzy aktualnym stanem/pomiarem a~pomiarem sygnału sterującego,
	\item $w$ --  zmienna losowa o~rozkładzie normalnym będąca szumem procesu,
	\item $Q$ --  kowariancja szumu procesu.
\end{itemize}

Efektem działania tej fazy jest oszacowana wartość aktualnego stanu obserwowanego źródła wraz z~kowariancją tego oszacowania.

Drugim etapem jaki możemy wyszczególnić w~trakcie działania omawianego filtru jest faza korekcji lub inaczej aktualizacji pomiarowej. Matematycznie fazę tę można zapisać za pomocą równań \ref{eq:appx:KF:ph2a}, \ref{eq:appx:KF:ph2b}, \ref{eq:appx:KF:ph2c} oraz \ref{eq:appx:KF:ph2d}.

\begin{subequations}
	\begin{align}
		z_k           & = H*\widehat{x}^-_k + v_k \label{eq:appx:KF:ph2a}                        \\
		K_k           & = P^-_{k}H^T(HP^-_{k}H^T + R)^{-1} \label{eq:appx:KF:ph2b}               \\
		\widehat{x}_k & = \widehat{x}^-_k + K_{k}(z_k-H\widehat{x}^-_k)  \label{eq:appx:KF:ph2c} \\
		P_k           & = (I-K_{k}H)P^-_k \label{eq:appx:KF:ph2d}                                
	\end{align}
\end{subequations}

gdzie:
\begin{itemize}
	\item $z_k$ -- aktualny pomiar uzyskany ze źródła w kroku $k$,
	\item $H$ -- macierz definiująca powiązanie aktualnego pomiaru z~oszacowanym stanem,
	\item $v_k$ -- szum pomiaru,
	\item $K_k$ -- wzmocnienie Kalmana (\emph{ang. Kalman gain}),
	\item $R$ -- macierz wariancji uzyskanego pomiaru,
	\item $I$ -- macierz jednostkowa.
\end{itemize}

Wartością, którą filtr zwraca jako wynik swojego działania jest $\widehat{x}_k$.

Obie fazy wykonywane są cyklicznie, a~kompletny algorytm przedstawia schemat z rysunku \ref{fig:appx:KF:algorithm}.

\begin{savenotes}
	\begin{figure}[!htb]
		\centering
		\begin{tikzpicture}[>=latex']
			\tikzset{block/.style= {draw, rectangle, align=center,minimum width=4cm,minimum height=1cm},
				rblock/.style={draw, shape=rectangle,rounded corners=1.5em,align=center,minimum width=2cm,minimum height=1cm},
				input/.style={ % requires library shapes.geometric
					draw,
					trapezium,
					trapezium left angle=60,
					trapezium right angle=120,
					minimum width=2cm,
					align=center,
					minimum height=1cm
				},
			}
			\node [block]    (predicate)      
			{  
				\begin{tabular}{l}
					\textbf{Faza predykacji}                                          \\
					\parbox{5cm}{                                                     
					\begin{itemize}[leftmargin=*]                                     
					\item Estymacja aktualnego stanu \ref{eq:appx:KF:ph1a}            \\
					\item Estymacja kowariancji aktulnego stanu \ref{eq:appx:KF:ph1b} \\
					\end{itemize}}                                                    
				\end{tabular} 
			};
																																																
			\node [block, right =1cm of predicate] (correction)      
			{  
				\begin{tabular}{l}
					\textbf{Faza korekcji}                                                            \\
					\parbox{6.5cm}{                                                                   
					\begin{itemize}[leftmargin=*]                                                     
					\item {Wyznaczenie współczynnika wzmocnienia Kalmana \ref{eq:appx:KF:ph2a}}     \\
					\item {Korekta estymacji stanu w~oparciu o~uzyskany pomiar \ref{eq:appx:KF:ph2b}} \\
					\item {Obliczenie kowariancji błędu estymacji \ref{eq:appx:KF:ph2c}}            \\
					\end{itemize}}                                                                    
				\end{tabular} 
			};    
			%% paths
			\path[every node/.style={font=\sffamily\small,
				fill=white,inner sep=1pt}]
			(predicate.north) edge [->, very thick, bend left] (correction.north)
			(correction.south) edge [->, very thick, bend left] (predicate.south);
		\end{tikzpicture}
		\caption{Schemat działania filtru Kalmana}
		\label{fig:appx:KF:algorithm}
	\end{figure}
\end{savenotes}
		
Filtr Kalmana, dzięki swojej budowie, może być używany nie tylko w~procesie odszumiania sygnału. Drugim ważnym i~popularnym jego zastosowaniem jest łączenie ze sobą sygnałów z~różnych źródeł. Wówczas wartości jednego z~nich wykorzystywane są w~fazie predykacji, a~drugiego w~fazie korekcji.
		
		
\section*{Rozszerzony filtr Kalmana (EKF - \emph{ang. Extended Kalman Filter})}
\label{sec:appx:filters:EKF}
\nomenclature{EKF}{Rozszerzony filtr Kalmana}
Filtr Kalmana w~swojej pierwotnej formie stanowi optymalny estymator dla procesów o~charakterze liniowym. Filtracja sygnałów nieliniowych wymagała modyfikacji oryginalnej formuły i~prace takie zostały podjęte niemal natychmiast po opublikowaniu prac Rudolpha Kalmana. W~1962 Stanley F. Schmidt, jeden z~dyrektorów NASA zaangażowany w~program Apollo, wraz z~zespołem opublikował pracę zawierającą zmodyfikowaną formułę liniowego filtru Kalmana przetwarzającą dane nieliniowe \cite{smith1962application}. Działanie filtru opiera się na linearyzacji przetwarzanego sygnału za pomocą szeregu Taylora. Schemat działania jest analogiczny do tego przedstawionego na rysunku \ref{fig:appx:KF:algorithm}, jednak zmianie ulegają poszczególne równania. Faza estymacji opiera się na równaniach \ref{eq:appx:EKF:ph1a} i~\ref{eq:appx:EKF:ph1b}.
		
\begin{subequations}
	\begin{align}
		\widehat{x}^-_k & = f(\widehat{x}_{k-1}, u_{k-1}, w_{k-1})\label{eq:appx:EKF:ph1a} \\
		P^-_k           & = A^J P_{k-1} {A^J}^T + W^J Q {W^J}^T \label{eq:appx:EKF:ph1b}   
	\end{align}
\end{subequations}
		
gdzie:
\begin{itemize}
	\item $f()$ -- funkcja nieliniowa określająca przejście pomiędzy kolejnymi stanami,
	\item $A^J , W^J$ -- jakobiany funkcji $f()$.
\end{itemize}
		
Równania \ref{eq:appx:EKF:ph2a}, \ref{eq:appx:EKF:ph2b}, \ref{eq:appx:KF:ph2c} oraz \ref{eq:appx:KF:ph2d} definiują natomiast fazę korekcji.
		
\begin{subequations}
	\begin{align}
		z_k           & = h(\widehat{x}^-_k, v_k) \label{eq:appx:EKF:ph2a}                                 \\
		K_k           & = P^-_{k}{H^J}^T(H^J P^-_{k}{H^J}^T + V^J R {V^J}^T)^{-1} \label{eq:appx:EKF:ph2b} \\
		\widehat{x}_k & = \widehat{x}^-_k + K_{k}(z_k-h(\widehat{x}^-_k, v_k))  \label{eq:appx:EKF:ph2c}   \\
		P_k           & = (I-K_{k}H^J)P^-_k \label{eq:appx:EKF:ph2d}                                       
	\end{align}
\end{subequations}
		
gdzie:
\begin{itemize}
	\item $h$ -- funkcja nieliniowa definiująca relację pomiędzy estymowanym stanem a~faktycznym pomiarem,
	\item $H^J, V^J$ -- jakobiany funkcji $h$.
\end{itemize}
		
Sposób wyznaczenia użytych jakobianów prezentują równania \ref{eq:appx:EKF:J1}, \ref{eq:appx:EKF:J2}, \ref{eq:appx:EKF:J3} oraz \ref{eq:appx:EKF:J4}.
		
\begin{subequations}
	\begin{align}
		A^J_{[i, j]} & = \frac{\partial f_{[i]}}{\partial x_{[j]}}(\widehat{x}_{k-1}, u_{k-1}, w_{k-1}) \label{eq:appx:EKF:J1} \\
		W^J_{[i, j]} & = \frac{\partial f_{[i]}}{\partial w_{[j]}}(\widehat{x}_{k-1}, u_{k-1}, w_{k-1}) \label{eq:appx:EKF:J2} \\
		H^J_{[i, j]} & = \frac{\partial h_{[i]}}{\partial x_{[j]}}(\widehat{x}^-_k, v_k) \label{eq:appx:EKF:J3}                \\
		V^J_{[i, j]} & = \frac{\partial h_{[i]}}{\partial v_{[j]}}(\widehat{x}^-_k, v_k) \label{eq:appx:EKF:J4}                
	\end{align}
\end{subequations}
		
Filtr EKF okazał się skuteczny dla filtracji sygnałów nieliniowych jednak konieczność obliczania jakobianów zwiększa koszty obliczeniowe tego rozwiązania, co jest szczególnie widoczne w~przypadku częstego dokonywania pomiarów. 
		
		
\section*{Bezśladowy filtr Kalmana (UKF - \emph{ang. Unscented Kalman Filter})}
\label{sec:appx:filters:UKF}
\nomenclature{UKF}{Bezśladowy filtr Kalmana}
W 1995 w~wyniku prac prowadzonych przez S. Juliera, J. Uhlmanna i~H. Durrant-Whyte\'a została opublikowana \cite{Julier1995} metoda filtrowania sygnałów nieliniowych oparta o~analizę danych statystycznych filtrowanego sygnału. Realizowana jest ona poprzez zastosowanie przekształcenia bezśladowego operującego na rozkładzie prawdopodobieństwa filtrowanego sygnału, ponieważ mniej kosztowne, a~jednocześnie dokładniejsze jest aproksymowanie takich danych, niż funkcji nieliniowych \cite{Uhlmann94}.
Względem EKF, w~przypadku tego algorytmu, występuje dodatkowy krok -- wyznaczenie punktów sigma (równanie \ref{eq:appx:UKF:ph0}). 
		
\begin{equation}
	\chi^\alpha_{k-1} = [\widehat{x}^\alpha_{k-1} \widehat{x}^\alpha_{k-1}\pm\sqrt{(L+\lambda)P^\alpha_{k-1}}] \label{eq:appx:UKF:ph0}
\end{equation}
		
Kolejnym krokiem jest faza predykacji wyrażona za pomocą wzorów \ref{eq:appx:UKF:ph1a} - \ref{eq:appx:UKF:ph1e}
\begin{subequations}
	\begin{align}
		\chi^x_{k}        & = F[\chi^x_{k-1},\chi^v_{k-1}] \label{eq:appx:UKF:ph1a}                                                               \\
		\widehat{x}^-_{k} & = \sum_{i=0}^{2L}W_i^{(m)}\chi^x_{i,k}\label{eq:appx:UKF:ph1b}                                                        \\
		P^-_k             & = \sum_{i=0}^{2L}W_i^{(c)}[\chi^x_{i,k}-\widehat{x}^-_{k}][\chi^x_{i,k}-\widehat{x}^-_{k}]^T \label{eq:appx:UKF:ph1c} \\
		\gamma_k          & = H[\chi^x_k, \chi^n_{k-1}] \label{eq:appx:UKF:ph1d}                                                                  \\
		\widehat{y}^-_{k} & = \sum_{i=0}^{2L}W_i^{(m)}\gamma_{i,k} \label{eq:appx:UKF:ph1e}                                                       
	\end{align}
\end{subequations}
		
Faza korekcji została wyrażona wzorami \ref{eq:appx:UKF:ph2a} - \ref{eq:appx:UKF:ph2e}
\begin{subequations}
	\begin{align}
		P_{\widehat{y}_k\widehat{y}_k} & = \sum_{i=0}^{2L}W_i^{(c)}[\gamma_{i,k} - \widehat{y}^-_k][\gamma_{i,k} - \widehat{y}^-_k]^T\label{eq:appx:UKF:ph2a} \\
		P_{\widehat{x}_k\widehat{y}_k} & = \sum_{i=0}^{2L}W_i^{(c)}[\chi_{i,k} - \widehat{x}^-_k][\gamma_{i,k} - \widehat{y}^-_k]^T\label{eq:appx:UKF:ph2b}   \\
		K_k                            & = P_{\widehat{x}_k\widehat{y}_k} P^{-1}_{\widehat{y}_k\widehat{y}_k} \label{eq:appx:UKF:ph2c}                        \\
		\widehat{x}_k                  & = \widehat{x}^-_{k} + K_k(y_k - \widehat(y)^-_k) \label{eq:appx:UKF:ph2d}                                            \\
		P_k                            & = P^-_k - K_kP_{\widehat{y}_k\widehat{y}_k}K_k^T \label{eq:appx:UKF:ph2e}                                            
	\end{align}
\end{subequations}
		
%gdzie:
%\begin{itemize}
%	\item TODO
%	\item TODO
%	\item TODO
%	\item TODO
%\end{itemize}
		
Dokładny opis bezśladowego filtru Kalmana można znaleźć między innymi w~pracach E. Wana oraz R. van der Merwe \cite{Wan2000, Wan2001}, którzy pracowali nad rozwinięciem tej metody. Wspomniani autorzy, w~tej samej pracy, porównali ze sobą dokładność szacowania dla EKF i~UKF wykorzystując sygnał wygenerowany na podstawie równania Mackey-Glassa \cite{Glass2010} zaburzonego białym szumem. Zamieszczone wyniki jednoznacznie pokazują, że filtr UKF był w~stanie oszacować odfiltrowany sygnał z~mniejszym błędem niż EKF. Podobne badania porównujące oba te filtry były prowadzone także w~polskich ośrodkach badawczych między innymi na Wojskowej Akademii Technicznej. Także i~w tym przypadku, opublikowane wyniki wskazywały na większą dokładność filtru UKF \cite{Konatowski2007, Konatowski2007a}.
		
\section*{Filtr komplementarny}\label{sec:appx:filters:CF}
Filtr Kalmana w~swojej podstawowej postaci, jak i~w wersji rozszerzonej został od razu wykorzystany w~pozaakademickim środowisku i~stał się niejako standardowym narzędziem filtracji i~fuzji sygnałów. Jednak w~przypadku kiedy stan obserwowanego systemu jest wyrażony przez więcej niż jedną wartość liczbową, implementacja filtru Kalmana wymaga wykorzystania implementacji algebry macierzowej, która jest złożona obliczeniowo \cite{wiki:MatrixAlgebraComplexity2016}. Świadomość tego ograniczenia skłoniła badaczy do poszukiwania alternatywnego sposobu na filtrację sygnałów przy zachowaniu zbliżonej dokładności. Przykład efektów takich badań możemy zobaczyć w~pracy Higginsa \cite{Higgins1975}, w~której prezentuje on postać filtru komplementarnego oraz wykazuje związek takiej metody z~filtrem Kalmana. Typowym zastosowaniem filtru komplementarnego jest złączenie ze sobą sygnałów opisujących to samo zjawisko, ale odznaczających się różną częstotliwością szumu (wysoką i~niską). Ogólny schemat podstawowego filtru komplementarnego zaprezentowany został na rysunku \ref{fig:appx:CF:algorithmA}.
		
% Defining string as labels of certain blocks.
\newcommand{\suma}{\Large$+$}
\newcommand{\inte}{$\displaystyle \int$}
\newcommand{\derv}{\huge$\frac{d}{dt}$}
		
		
\begin{savenotes}
	\begin{figure}[!htb]
		\centering 
		\begin{subfigure}[b]{0.49\textwidth}
			\centering
			\begin{tikzpicture}[auto, thick, node distance=2cm, >=triangle 45]
				\tikzset{%
					block/.style    = {draw, thick, rectangle, minimum height = 3em,
						minimum width = 3em},
					sum/.style      = {draw, circle, node distance = 2cm}, % Adder
					input/.style    = {coordinate}, % Input
					output/.style   = {coordinate} % Output
				}
																																																																							
				\draw
				node at (0,0)[right=-3mm]{\Large \textopenbullet}
				node [input, name=xSignal] {} 
				node at (4,0) [block] (lpf) {$G(s)$}
																																																																							
				node at (0,-2)[right=-3mm]{\Large \textopenbullet}
				node [input, below of=xSignal, name=ySignal] {} 
				node at (4,-2)[block] (hpf) {$1-G(s)$}
																																																																							
				node at (7, -1) [sum, name=suma]{\suma}
																																																																							
				node [output, right of=suma](out) {} ;
																																																																							
				\draw[->](xSignal) -- node {$x = z~+ n_1$}(lpf);
				\draw[->](ySignal) -- node {$y = z~+ n_2$}(hpf);
				\draw[->](hpf) -| node {}(suma);
				\draw[->](lpf) -| node {}(suma);
				\draw[->](suma) -- node {$\widehat{z}$}(out);
			\end{tikzpicture}
			\caption{Wariant podstawowy}
			\label{fig:appx:CF:algorithmA}			
		\end{subfigure}
																					
		\begin{subfigure}[b]{0.49\textwidth}
			\centering
			\begin{tikzpicture}[auto, thick, node distance=2cm, >=triangle 45]
				\tikzset{%
					block/.style    = {draw, thick, rectangle, minimum height = 3em,
						minimum width = 3em},
					sum/.style      = {draw, circle, node distance = 2cm}, % Adder
					input/.style    = {coordinate}, % Input
					output/.style   = {coordinate} % Output
				}
																																																																							
				\draw
				node at (0,0)[right=-3mm]{\Large \textopenbullet}
				node [input, name=xSignal] {} 		
				node at (3,0)[right=-2.5mm](point){\Large \textbullet}
																																																																							
				node at (0,-2)[right=-3mm]{\Large \textopenbullet}		
				node [input, below of=xSignal, name=ySignal] {} 
				node at (3, -2) [sum, name=sum1]{$-$}
				node at (6,-2)[block] (lpf) {$G(s)$}
																																																																							
				node at (8, -1) [sum, name=sum2]{\suma}
																																																																							
				node [output, right of=sum2](out) {} ;
																																																																							
				\draw[->](xSignal) -- node {$x = z~+ n_1$}(point);
				\draw[->](point) -- node {}(sum1);
				\draw[->](point) -| node {}(sum2);
				\draw[->](ySignal) -- node {$y = z~+ n_2$}(sum1);
				\draw[->](sum1) -- node {$n_2-n_1$}(lpf);		
				\draw[->](lpf) -| node {}(sum2);
																																																																							
				\draw[->](sum2) -- node {$\widehat{z}$}(out);
			\end{tikzpicture}
			\caption{Wariant pracujący na szumie}
			\label{fig:appx:CF:algorithmB}			
		\end{subfigure}
		\caption{Schemat działania filtru komplementarnego}
		\label{fig:appx:CF:algorithm}					
	\end{figure}
\end{savenotes}
				
gdzie:
\begin{itemize}
	\item $x, y$ -- zaszumione sygnały wejściowe,
	\item $z$ -- sygnał nieposiadający szumu (idealny),
	\item $n_1, n_2$ -- szum jakiemu poddany jest sygnał $z$,
	\item $G(s)$ -- filtr dolnoprzepustowy; $G(s) = \frac{1}{{\tau}s+1}$,
	\item $1 - G(s)$ -- filtr górnoprzepustowy $1 - G(s) = 1 - \frac{1}{{\tau}s+1} = \frac{{\tau}s}{{\tau}s+1}$,
	\item $\widehat{z}$ -- sygnał wyjściowy będący odfiltrowanym złączeniem sygnałów $x$ i~$y$.
\end{itemize}
				
Zarówno wspomniany już Higgins, jak i~Brown \cite{BROWN1972} w~swoich artykułach pokazali związek pomiędzy filtrem Kalmana i~filtrem komplementarnym. Wykazali tym samym, że może on stanowić mniej złożoną i~mniej kosztowną obliczeniowo alternatywę dla filtru Kalmana bez znaczącej utraty jakości uzyskanego wyniku. Zagadnienie, w~kontekście którego odbywały się badania nad tym filtrem, obejmowało problem integracji sygnałów z~urządzeń pozwalających wyznaczyć prędkości oraz kierunek ruchu obiektu (w tym przypadku chodziło o~poruszanie się pojazdów kosmicznych).
\section*{Filtry Mahoney'a i~Madgwicka}
Podobnie jak w~przypadku filtru Kalmana, istnieje także wariant filtra komplementarnego dla sygnałów nieliniowych. W~podobnym kontekście prace nad opracowaniem takiego filtra prowadził między innymi Robert Mahoney, który zaprezentował serię artykułów przedstawiających zarówno sam filtr, jak i~sposób jego projektowania oraz zastosowanie \cite{Baldwin2007,Mahony2005a,Mahony2008,Euston2008}. 
Na bazie tych prac w~2010 roku Sebastian Madgwick opracował autorską metodę \cite{Madgwick2010,Madgwick2011} integracji danych z~czujników inercyjnych i~magnetycznych w~celu wyznaczenia orientacji w~przestrzeni obserwowanego obiektu. Algorytm ten według opublikowanych wyników eksperymentów wykazuje większą dokładność niż algorytm oparty o~filtr Kalmana. Analizując budowę tego filtra oraz schemat zamieszczony w~publikacji autora warto zauważyć jego analogię do schematu filtra komplementarnego (rys. \ref{fig:appx:CF:algorithmA}).
%Rysunek \ref{fig:appx:Madg:algorithm} przedstawia schemat działania filtru Madgwicka. Warto zauważyć, że analogię w~jego budowie do schematu filtra komplementarnego \ref{fig:appx:CF:algorithmA}.
				
%\begin{savenotes}
%\begin{figure}[!htb]
%	\centering 
					
%		\begin{tikzpicture}[auto, thick, node distance=2cm, >=triangle 45]
%		\tikzset{%
%			block/.style    = {draw, thick, rectangle, minimum height = 3em,
%				minimum width = 3em},
%			sum/.style      = {draw, circle, node distance = 2cm}, % Adder
%			input/.style    = {coordinate}, % Input
%			output/.style   = {coordinate} % Output
%		}
							
%		\draw
%		node at (0,0)[right=-3mm]{\Large \textopenbullet}
%		node [input, name=xSignal] {} 
%		node at (4,0) [block] (lpf) {$G(s)$}
%		
%		node at (0,-2)[right=-3mm]{\Large \textopenbullet}
%		node [input, below of=xSignal, name=ySignal] {} 
%		node at (4,-2)[block] (hpf) {$1-G(s)$}
%		
%		node at (7, -1) [sum, name=suma]{\suma}
%		
%		node [output, right of=suma](out) {} ;
%		
%		\draw[->](xSignal) -- node {$x = z~+ n_1$}(lpf);
%		\draw[->](ySignal) -- node {$y = z~+ n_2$}(hpf);
%		\draw[->](hpf) -| node {}(suma);
%		\draw[->](lpf) -| node {}(suma);
%		\draw[->](suma) -- node {$\widehat{z}$}(out);
%		\end{tikzpicture}
%		\label{fig:appx:Madg:algorithm}
%
%	\end{figure}
%	\end{savenotes}
