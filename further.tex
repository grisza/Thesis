\cleardoublepage
\phantomsection
\chapter{Dalsze prace ([TODO] ja bym rozdziału nie nazywał dalsze prace tylko wszystko jako podsumowanie, szczególnie że oba rozdziały z części 5 są krótkie. Poza tym nie politycznie jest zaczynać podsumowanie od tego co można jeszcze zrobić. Trzeba do bólu podkreślać co się zrobiło i jakie to może miec znaczenie dla ludzkośći)}\label{chap:durther}
Wyniki uzyskane za pomocą metody przedstawionej w~niniejszej pracy wykazują przewagę opracowanego rozwiązania nad innymi metodami opisanymi współcześnie w literaturze. ([TODO] Nie można rozpoczynać podsumowania od tego co będzie zrobione. Trzeba podkreślać to co zostało zrobione) W~perspektywie dalszych badań celowe jest sprawdzenie i~ewentualne dostosowanie metody do danych pochodzących z~drugiej wersji Kinecta działającego w~oparciu o~inną technologię niż urządzenie użyte w~tej pracy. Odmienność tego urządzenia może sprawić, że część aktualnie wykonywanych kroków może być pominięta lub zastąpiona innymi. \\

Kolejnym kierunkiem badań jaki można podjąć to dostosowanie algorytmu do śledzenia innych części ciała ([TODO] to jest strzałw stopę, bo metoda musi być uniwerslana dla każdej kończyny inaczej nie jest metodą). Ze względu na czas prowadzonych badań te skupiły się na dostosowaniu metody do ruchu ręki ([TODO] nie można tak pisać). Mając świadomość specyfiki ruchu każdej z~kończyn: szybkości wykonywanego ruchu, wielkości samej kończyny czy zakresu w~jakim się ona porusza, metoda może wymagać przedefiniowania warunków, dla których zebrane dane uznajemy za wiarygodne.([TODO] takiego czegoś bym nie pisał. Bezpiecznie jest wspomnieć o szybkości ruchu, w którym dane będą znacznie bardziej pofragmentowane, ale ich wiarygodność nie powinna się zmienić)\\

Z punktu widzenia prowadzonych ([TODO] kto to jest prowadzący?) w~trakcie działania metody obliczeń zasadnym wydaje się przeprowadzenie dodatkowych badań związanych z~bezpośrednim łączeniem ze sobą kwaternionów w~taki sposób aby można było nadać różne wartości wag dla każdej z~osi obrotu ([TODO] Wpisując tak oczywiste i relatywnie krótkotrwałe rozszerzenia dotychczasoych osiągnąć ktoś może łątwo zapytać a dlaczego to nie zostało zrobione, skoro jest oczywiste. Swoją drogą warto to może kiedyś zrobić.). Zastosowana w~niniejszej pracy transformacja do postaci kątów Eulera, choć czytelna dla programisty, z~punktu widzenia prowadzonych obliczeń może być nie efektywna. Hipotetycznie jej użycie może doprowadzić do problemu powstania niejednoznaczności obrotu (\emph{ang. gimbal lock}) jednak w~trakcie eksperymentów do takiej sytuacji nie doszło ([TODO] to znowu kopanie się po kostkach).\\

Duże znaczenie dla przeprowadzonych badań miały warunki, zwłaszcza oświetleniowe, jakie panowały w~laboratorium. Zdarzyło się kilkukrotnie, że ćwiczenia musiały zostać przerwane, bądź to ze względu na unoszenie się zanieczyszczenia niewiadomego pochodzenia w~powietrzu, które odbijało światło w~sposób zakłócający działanie Kinecta, bądź też ze względu na zakłócenia jakim poddana była transmisja danych za pomocą Bluetooth. Objawiało się to tym, że do komputera, który odbierał i~rejestrował wszystkie dane, pakiety z~pomiarami IMU trafiały niekompletne lub uszkodzone. Elementami które wymagać mogą więc dalszego rozwoju jest np. poprawa formatu danych na taki, który byłby w~stanie je odtworzyć w~przypadku niewielkiego uszkodzenia w~trakcie transmisji ([TODO] Ten akapit mi się podoba bo definiuje problemy podczas badań i daleko idące usprawnienia mocno wykraczające poza główny aspekt pracy).\\

Wreszcie zasadnym wydaje się ([TODO] zasadne, czyli można przeczytać jako niezbędne. MOże jakoś mniej zobowiązująco, np. Kolejnym etapem badań będzie...) zweryfikowanie działania metody stosując inne, alternatywne kotrolery jak np.: ([TODO]trzeba konkretnie podać)(z~innymi urządzeniami tego samego rodzaju co te zaprezentowane w~niniejszej pracy). Szczególnie interesujący wydaje się być sposób działania ([TODO] czego?)z~wykorzystaniem analogowych IMU ponieważ mogą one wprowadzić konieczność dodatkowych kroków filtrujących te dane przed dalszymi obliczeniami.([TODO] ostatnie zdanie brzmiało, że jak zastosujemy analogowe IMU to będzie więcej roboty - to chyba nie tak miało brzmieć)\\

Ostatnim z~rozważanych przeze mnie kierunków dalszych badań jest modyfikacja urządzenia pobierającego ([TODO] czy agregujący?)  sygnały z~czujników IMU ([TODO] ale jaka modyfikacja). Jednym z~wniosków, jaki został wysunięty, na podstawie zakończonych już badań, była koncepcja zamiany jednego urządzenia wysyłającego pomiary przechwycone z~wielu czujników ([TODO] nie mogę sobie wyobrazić jednego urządzenia wysyłającego dane z wielu czujników?!!) na rzecz wielu autonomicznych modułów, z~których każdy zawiera swój zegar i~wysyła dane niezależnie od innych. Głównym powodem rozważanej modyfikacji wydaje się być przede wszystkim wygoda użytkowania całego systemu poprzez wyeliminowanie przewodów łączących markery z~jednostką centralną (Arduino). Z~drugiej jednak strony taka zmiana wymagałaby przeprowadzenia synchronizacji zegarów pomiędzy nawet kilkunastoma urządzeniami, a~nie tylko komputerem PC a~jednostką centralną urządzenia pomiarowego ([TODO] trochę to trzeba bardziej klarownie opisać).