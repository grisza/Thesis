\nomenclature{MoCap}{\emph{ang. Motion Capture} -- system śledzenia i~rejestrowania ruchu}
\nomenclature{GPS}{\emph{ang. Global Positioning System} -- globalny system pozycjonowania}
\nomenclature{LPM}{\emph{ang. Local Position Measurement System} -- lokalny system pozycjonowania}
\nomenclature{IMU}{\emph{ang. Inertial Measurement Unit} -- inercyjne jednostki pomiarowe: akcelerometr i~żyroskop}
\nomenclature{MARG}{\emph{ang. Magnetic, Angular Rate, and Gravity} -- inercyjne jednostki pomiarowe: akcelerometr i~żyroskop wspierane przez magnetometr}
\nomenclature{KF}{Liniowy filtr Kalmana}
\nomenclature{EKF}{\emph{ang. Extended Kalman Filter} -- Rozszerzony filtr Kalmana}
\nomenclature{UKF}{\emph{ang. Unscented Kalman Filter} -- Bezśladowy filtr Kalmana}
\nomenclature{$\alpha$}{kąt obrotu sylwetki użytkownika względem kontrolera Kinect}
\nomenclature{$\beta$}{kąt zgięcia ręki mierzony w~stawie łokciowym}
\nomenclature{$\phi$}{kąt Eulera określający obrót wokół osi X}
\nomenclature{$\theta$}{kąt Eulera określający obrót wokół osi Y}
\nomenclature{$\psi$}{kąt Eulera określający obrót wokół osi Z}
\nomenclature{$A = [a_x, a_y, a_z]$}{wartości pomiarów z~akcelerometru}
\nomenclature{$G = [g_x, g_y, g_z]$}{wartości pomiarów z~żyroskopu}
\nomenclature{$T$}{pomiar temperatury pracy modułu inercyjnego}
\nomenclature{$T_0$}{neutralna temperatura pracy modułu inercyjnego wg specyfikacji}
\nomenclature{$t$}{znacznik czasu otrzymanych pomiarów}
\nomenclature{$t_{szum}$}{czas, w którym pomiary z~kontrolera Kinect pozostają uznawane za niewiarygodne}
%\nomenclature{$\Delta t$}{upływ czasu pomiędzy kolejnymi pomiarami}
\nomenclature{$Q = [q_w, q_x, q_y, q_z]$}{orientacja przestrzenna w~postaci kwaternionowej}
\nomenclature{$\blacksquare^F$}{indeks oznaczający estymaty będące wynikiem fuzji danych}
\nomenclature{$\blacksquare^K$}{indeks oznaczający dane uzyskane z~kontrolera Kinect}
\nomenclature{$\blacksquare^I$}{indeks oznaczający dane uzyskane z~modułów inercyjnych}
\nomenclature{$P = [p_x, p_y, p_z]$}{położenie przestrzenne stawu}
\nomenclature{$E = [\phi, \theta, \psi]$}{orientacja przestrzenna w~postaci kątów Eulera}
\nomenclature{$f_T$}{współczynnik korekty temperatury}
\nomenclature{$f_{LPF}$}{współczynnik filtracji pomiarów z kontrolera Kinect, wykorzystywany w~filtrze LPF}
\nomenclature{$f_m$}{współczynnik filtracji pomiarów z czujników inercyjnych, wykorzystywany w filtrze Madgwicka}
\nomenclature{$w_\phi$}{waga wartości kąta Eulera $\phi$}
\nomenclature{$w_\theta$}{waga wartości kąta Eulera $\theta$}
\nomenclature{$w_\psi$}{waga wartości kąta Eulera $\psi$}
\nomenclature{$A_{th}=[a_x,a_y,a_z]_{th}$}{maksymalny dopuszczalny błąd spoczynkowych pomiarów akcelerometru}
\nomenclature{$G_{th}=[g_x,g_y,g_z]_{th}$}{maksymalny dopuszczalny błąd spoczynkowych pomiarów żyroskopu}
\nomenclature{$A_0=[a_x,a_y,a_z]_0$}{oczekiwane wartości spoczynkowe pomiarów akcelerometru}
\nomenclature{$G_0=[g_x,g_y,g_z]_0$}{oczekiwane wartości spoczynkowe pomiarów żyroskopu}
\nomenclature{$\blacksquare_{sh_L}$}{indeks oznaczający staw barkowy lewy \emph{ang. shoulder left}}
\nomenclature{$\blacksquare_{sh_R}$}{indeks oznaczający staw barkowy prawy \emph{ang. shoulder right}}
\nomenclature{$\blacksquare_e$}{indeks oznaczający staw łokciowy \emph{ang. elbow}}
\nomenclature{$\blacksquare_w$}{indeks oznaczający staw nadgarstkowy \emph{ang. wrist}}
\nomenclature{$\blacksquare_j$}{indeks oznaczający wybrany staw \emph{ang. joint}}
\nomenclature{$\blacksquare_{j-1}$}{indeks oznaczający staw nadrzędny w~hierarchicznym modelu szkieletowym}
\nomenclature{$m(A,G,f_m,\Delta t)$}{formuła filtru Madgwicka}
\nomenclature{$\bar{A} = [\bar{a}_x, \bar{a}_y, \bar{a}_z ]$}{uśrednione wartości pomiarów z~akcelerometru}
\nomenclature{$\bar{G} = [\bar{g}_x, \bar{g}_y, \bar{g}_z ]$}{uśrednione wartości pomiarów z~żyroskopu}
\nomenclature{$cor_A=[c_{ax},c_{ay},c_{az}]$}{współczynniki korekty pomiarów akcelerometru}
\nomenclature{$cor_G=[c_{gx},c_{gy},c_{gz}]$}{współczynniki korekty pomiarów żyroskopu}
\nomenclature{$cor=[cor_A,cor_G]$}{współczynniki korekty pomiarów modułów inercyjnych}
\nomenclature{$\tau$}{przesunięcie czasowe pomiędzy sygnałami}
\nomenclature{$\tau_{max}$}{przesunięcie czasowe pomiędzy sygnałami, dla którego korelacja pomiędzy nimi przyjmuje największą wartość}
\nomenclature{$I$}{sygnał, którego źródłem jest moduł inercyjny}
%\nomenclature{$I[t]$}{próbka sygnału, którego źródłem jest moduł inercyjny w~chwili t}
\nomenclature{$K$}{sygnał, którego źródłem jest Kinect}
%\nomenclature{$K[t]$}{próbka sygnału, którego źródłem jest Kinect w~chwili t}
\nomenclature{$I \ast K$}{korelacja wzajemna sygnałów o~źródłach w~module inercyjnym oraz kontrolerze Kinect}
\nomenclature{$A'$}{skorygowany pomiar akcelerometru ze względu na temperaturę}
\nomenclature{$P'$}{położenie przestrzenne stawu po filtracji}
\nomenclature{$T_{raw}$}{pomiar temperatury w~formacie bezpośrednio odczytanym z~modułu inercyjnego}
\nomenclature{$T_{deg}$}{pomiar temperatury przedstawiony w~formie stopni Celsjusza wyznaczony na podstawie $T_{raw}$}
\nomenclature{$g$}{jednostka przyspieszenia grawitacyjnego. W~przybliżeniu $1g = 9.80665 m/{s^2}$}
\nomenclature{$f_A$}{współczynnik konwersji bezpośrednich pomiarów akcelerometru do jednostek przyspieszenia grawitacyjnego}
\nomenclature{$f_G$}{współczynnik konwersji bezpośrednich pomiarów żyroskopu do prędkości kątowej $\degree/_s$}
\nomenclature{$\widetilde{\omega}$}{średnia wartość szumu błądzenia (ARW - \emph{ang. Angular Random Walk}) żyroskopu}
\nomenclature{$d_e (P_1,P_2)$}{Odległość euklidesowa pomiędzy dwoma dowolnymi punktami $P_1,P_2$ w~przestrzeni}
%\nomenclature{$HOMT$}{skrótowa nazwa autorskiej metody śledzenia ruchu opisywanej w~ninejszej pracy \emph{ang. Hybrid, Oriented-based Motion Tracking}}
