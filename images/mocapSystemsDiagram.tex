\begin{figure}[!htb]
	\centering
	\begin{tikzpicture}[
			basic/.style  = {draw, text width=5cm, drop shadow, font=\sffamily, rectangle},
			root/.style   = {basic, rounded corners=2pt, thin, align=center, fill=black!20}, 
			level 1/.style={sibling distance=40mm},
			level 2/.style = {basic, rounded corners=6pt, thin,align=center, fill=white, text width=8em},
			level 3/.style = {basic, thin, align=left, fill=pink!60, text width=6.5em},	
			edge from parent/.style={->,draw},
		>=latex]
		
		% root of the the initial tree, level 1
		\node[root] {Systemy śledzenia ruchu}
		% The first level, as children of the initial tree
		child {node[level 2, xshift=-30pt] (c1) {Optyczne}
			child{node[level 2] (c11) {Z markerami}}
			child{node[level 2] (c12) {Bez markerów}}
		}
		child {node[level 2, xshift=40pt] (c2) {Nie optyczne}};
		  
		
		%  child {node[level 2] (c3) {Drawing arrows between nodes}};
		
		% The second level, relatively positioned nodes
		\begin{scope}[every node/.style={level 2}]
			\node [below of = c11, xshift=10pt] (c111) {Aktywne};
			\node [below of = c111] (c112) {Pasywne};
			
			\node [below of = c12, xshift=10pt] (c121) {Aktywne};
			\node [below of = c121] (c122) {Pasywne};
			
			\node [below of = c2, xshift=10pt] (c21) {Inercyje};
			\node [below of = c21] (c22) {Magnetyczne};
			\node [below of = c22] (c23) {Elektromagnetyczne (Radiowe)};
			\node [below of = c23] (c24) {Akustyczne};
			\node [below of = c24] (c25) {Mechaniczne};
		\end{scope}
		
		
		% lines from each level 1 node to every one of its "children"
		\foreach \value in {1,2}
		\draw[->] (c11.185) |- (c11\value.west);
		
		\foreach \value in {1,2}
		\draw[->] (c12.185) |- (c12\value.west);
		
		
		\foreach \value in {1,...,5}
		\draw[->] (c2.180) |- (c2\value.west);
		
		%\foreach \value in {1,...,5}
		%   \draw[->] (c3.195) |- (c3\value.west);
	\end{tikzpicture}
	\caption{Podział systemów śledzenia ruchu ze względu na technologię pobierania danych}
	\label{fig:literature:mocapSystems:diagram}
\end{figure}