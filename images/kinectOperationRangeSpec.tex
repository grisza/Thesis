\begin{figure}
	\centering
	\subfigure[Horyzontalny zakres pracy]
	{
		\begin{tikzpicture}[scale=0.5]					
			\draw[very thin,fill=chartreuse(traditional)] (61.5:0.8) -- (61.5:4) arc (61.5:118.5:4) -- (118.5:0.8) arc (118.5:61.5:0.8) -- cycle;						
			\draw[very thin,fill=sepia] (61.5:4) -- (61.5:8) arc (61.5:118.5:8) -- (118.5:4) arc (118.5:61.5:4) -- cycle;						
			\draw[very thin,fill=gray80] (61.5:8) -- (61.5:9) arc (61.5:118.5:9) -- (118.5:8) arc (118.5:61.5:8) -- cycle;
																								
			\draw[dashed, very thin,fill=gray80] (61.5:8) -- (61.5:9) arc (61.5:0:9) -- (0:8) arc (0:61.5:8) -- cycle;
			\draw[dashed, very thin,fill=gray80] (61.5:4) -- (61.5:8) arc (61.5:0:8) -- (0:4) arc (0:61.5:4) -- cycle;
			\draw[dashed, very thin,fill=gray80] (61.5:0.8) -- (61.5:4) arc (61.5:0:4) -- (0:0.8) arc (0:61.5:0.8) -- cycle;
			\draw[dashed, very thin,fill=gray80] (61.5:0.4) -- (61.5:0.8) arc (61.5:0:0.8) -- (0:0.4) arc (0:61.5:0.4) -- cycle;
			\draw[dashed, very thin,fill=gray80] (61.5:0) -- (61.5:0.4) arc (61.5:0:0.4) -- (0:0) arc (0:61.5:0) -- cycle;
																								
			\draw[dashed, very thin,fill=gray80] (118.5:8) -- (118.5:9) arc (118.5:180:9) -- (180:8) arc (180:118.5:8) -- cycle;
			\draw[dashed, very thin,fill=gray80] (118.5:4) -- (118.5:8) arc (118.5:180:8) -- (180:4) arc (180:118.5:4) -- cycle;
			\draw[dashed, very thin,fill=gray80] (118.5:0.8) -- (118.5:4) arc (118.5:180:4) -- (180:0.8) arc (180:118.5:0.8) -- cycle;
			\draw[dashed, very thin,fill=gray80] (118.5:0.4) -- (118.5:0.8) arc (118.5:180:0.8) -- (180:0.4) arc (180:118.5:0.4) -- cycle;
			\draw[dashed, very thin,fill=gray80] (118.5:0) -- (118.5:0.4) arc (118.5:180:0.4) -- (180:0) arc (180:118.5:0) -- cycle;
																								
																								
			\draw[very thin,fill=gray80] (0:0) -- (61.5:0.4) arc (61.5:118.5:0.4) -- (0:0);
																																					
			\draw[fill=black] (0, 0) circle (0.15);
																																						
			\draw (0:0.4) node[below right, rotate=-45] {$0.4m$};
			\draw (0:1.2) node[below right, rotate=-45] {$0.8m$};
			\draw (0:4) node[below right, rotate=-45] {$4m$};		
			\draw (0:8) node[below right, rotate=-45] {$8m$};			
			\draw (0:9) node[below right, rotate=-45] {$9m$};
																																											
			\draw[<->, thin, white] (65:6) arc (65:115:6);
			\draw[white] (80.5: 6) node[above] {$57^\circ$};
																																				
			\draw (61.5: 9.5) node {$28.5^\circ$};
			\draw (118.5: 9.5) node {$28.5^\circ$};
			\draw[thick, ->] (90:0) -- (90:9.5);
																																								
			\draw (90: 10) node {Kierunek obserwacji};
		\end{tikzpicture}
	}
	\subfigure[Wertykalny zakres pracy]
	{
		\begin{tikzpicture}[scale=0.5]	
																																			
			\draw[very thin,fill=chartreuse(traditional)] (-21.5:0.8) -- (-21.5:4) arc (-21.5:21.5:4) -- (21.5:0.8) arc (21.5:-21.5:0.8) -- cycle;
			\draw[very thin,fill=sepia] (-21.5:4) -- (-21.5:8) arc (-21.5:21.5:8) -- (21.5:4) arc (21.5:-21.5:4) -- cycle;
			\draw[very thin,fill=gray80] (-21.5:8) -- (-21.5:9) arc (-21.5:21.5:9) -- (21.5:8) arc (21.5:-21.5:8) -- cycle;
			\draw[very thin,fill=gray80] (-21.5:0) -- (-21.5:0.4) arc (-21.5:21.5:0.4) -- (21.5:0) arc (21.5:-21.5:0) -- cycle;
			\draw[very thin,fill=white] (-21.5:0.4) -- (-21.5:0.8) arc (-21.5:21.5:0.8) -- (21.5:0.4) arc (21.5:-21.5:0.4) -- cycle;
																																			
			\draw[fill=black] (0, 0) circle (0.15);
			\draw[thick, ->] (0:0) -- (0:9.5);
			\draw (0: 10) node[right] {Kierunek obserwacji};
																																			
			\draw[<->, thin, white] (-18:4.3) arc (-18:18:4.3);
			\draw[white] (-15: 4.3) node[right] {$43^\circ$};
																																			
			\draw[->, thin, white, dashed] (0:5.5) arc (0:-10:5.5);
			\draw[white] (-10: 5.5) node[right] {$-21.5^\circ$};
																																			
			\draw[->, thin, white, dashed] (0:5.5) arc (0:10:5.5);
			\draw[white] (10: 5.5) node[right] {$+21.5^\circ$};
																									
			\draw (0:0) node[left] {Kinect}		;				
			\draw (20.5:0.4) node[above] {$0.4m$};
			\draw (33.5:1.2) node[above] {$0.8m$};
			\draw (21.5:4) node[above] {$4m$};		
			\draw (21.5:8) node[above] {$8m$};			
			\draw (21.5:9) node[above] {$9m$};
																																			
		\end{tikzpicture}
	}
	\subfigure{
		\begin{tikzpicture}[scale=0.6]	
			\node[draw=black,thick,rounded corners=2pt,below left=2mm] {%
				\begin{tabular}{@{}r@{ }l@{}}
					\raisebox{2pt}{\tikz{\draw[gray80, fill=gray80] (0,0) rectangle (5mm,2mm);}}                                   & Niewidoczne                       \\
					\raisebox{2pt}{\tikz{\draw[sepia, fill=sepia] (0,0) rectangle (5mm,2mm);}}                                     & Za daleko                         \\
					\raisebox{2pt}{\tikz{\draw[black, fill=white] (0,0) rectangle (5mm,2mm);}}                                     & Za blisko                         \\
					\raisebox{2pt}{\tikz{\draw[chartreuse(traditional), fill=chartreuse(traditional)] (0,0) rectangle (5mm,2mm);}} & Obszar roboczy                    \\
					\raisebox{2pt}{\tikz{\draw[<->, thin] (0:0) arc (0:180:0.3);}}                                                 & Zakres pracy                      \\
					\raisebox{2pt}{\tikz{\draw[<-, thin, dashed] (0:0) arc (0:180:0.3);}}                                          & Przesunięcie kierunku obserwacji \\
					\raisebox{2pt}{\tikz{\draw[fill=black] (0, 0) circle (0.15);}}                                                 & Kinect                            
				\end{tabular}};	
		\end{tikzpicture}		
	}
	\caption{Zakres pracy kontrolera Microsoft Kinect (na podstawie \cite{kinectSpec2016})}
	\label{fig:characteristics:kinect:range}
\end{figure}