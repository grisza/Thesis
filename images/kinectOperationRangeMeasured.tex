\begin{savenotes}
	\begin{figure}[H]
		\centering
		\subfigure
		{
			\begin{tikzpicture}[scale=0.8]
																																																														
				\shade[bottom color=gray80,top color=gray80] (0,0) rectangle (8.6,0.4);
				\shade[bottom color=white,top color=white] (0,0.4) rectangle (8.6,0.8);
				\shade[bottom color=blue,top color=chartreuse(traditional)] (0,0.8) rectangle (8.6,2);
				\shade[bottom color=chartreuse(traditional),top color=chartreuse(traditional)] (0,2) rectangle (8.6,2.3);
				\shade[bottom color=chartreuse(traditional),top color=red] (0,2.3) rectangle (8.6,3.5);
				\shade[bottom color=sepia,top color=sepia] (0,3.5) rectangle (8.6,5);
																																																													
																																																													
				\draw[fill=gray80] (4.3,0) -- (8.6,0) -- (8.6,5) -- (7,5) -- cycle;
				\draw[fill=gray80] (4.3,0) -- (0,0) -- (0,5) -- (1.6,5) -- cycle;
																																																													
																																																													
			\end{tikzpicture}
		}	
		\subfigure{
			\begin{tikzpicture}[scale=0.6]	
				\node[draw=black,thick,rounded corners=2pt,below left=2mm] {%
					\begin{tabular}{@{}r@{ }l@{}}
						\raisebox{2pt}{
						\tikz{
						\draw[gray80, fill=gray80] (0,0) rectangle (5mm,2mm);
						}
						}                                                                     & Niewidoczne                           \\
						\raisebox{2pt}{
						\tikz{
						\draw[sepia, fill=sepia] (0,0) rectangle (5mm,2mm);}}                 & Za daleko                             \\
						\raisebox{2pt}{\tikz{\draw[black, fill=white] (0,0) rectangle (5mm,2mm);
						}
						}                                                                     & Za blisko                             \\
																																																																																																			
						\raisebox{2pt}{	
						\tikz{
						\shade[bottom color=blue,top color=chartreuse(traditional),] (0,0) rectangle (5mm,2);
						\draw (6mm,0)   node[right] {$-0.06m$};
						\shade[bottom color=chartreuse(traditional),top color=chartreuse(traditional)] (0,2) rectangle (5mm,2.3);
						\draw (6mm,2.15) node[right] {$0m$};
						\shade[bottom color=chartreuse(traditional),top color=red] (0,2.3) rectangle (5mm,3.5);
						\draw (6mm,3.4) node[right] {$0.12m$};				
						}
						}                                                                     & Obszar roboczy z~wyznaczonym błędem \\
																																																																																																			
						\raisebox{2pt}{\tikz{\draw[<->, thin] (0:0) arc (0:180:0.3);}}        & Zakres pracy                          \\
						\raisebox{2pt}{\tikz{\draw[<-, thin, dashed] (0:0) arc (0:180:0.3);}} & Przesunięcie kierunku obserwacji     \\
						\raisebox{2pt}{\tikz{\draw[fill=black] (0, 0) circle (0.1);}}         & Kinect                                
					\end{tabular}};	
			\end{tikzpicture}		
		}
		\caption{Wykres zakresu pracy kontrolera Microsoft Kinect w~płaszczyźnie horyzontalnej z~uwzględnieniem niedokładności pomiaru odległości.}
		\label{fig:characteristics:kinect:rangeMeasured}
	\end{figure}
\end{savenotes}
		
