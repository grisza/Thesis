\begin{tikzpicture} [
		scale=0.8,
		auto,
		decision/.style = { diamond, draw=black, thick, fill=black!20,
			text width=5em, text badly centered,
			inner sep=1pt, rounded corners },
		block/.style    = { rectangle, draw=black, thick, 
			fill=black!20, text width=13em, text centered,
			rounded corners, minimum height=2em },
		line/.style     = { draw, thick, ->, shorten >=2pt },
	]
	% Define nodes in a~matrix
	\begin{scope} 
		\node [block] (block1) {Inicjalizacja czujników inercyjnych}; \\                    
		\node [block, below=of block1] (block2) {Inicjalizacja filtrów łączących dane z~akcelerometru i~żyroskopu}; \\
		\node [block, below=of block2] (block3) {Odszumienie danych z~czujników inercyjnych oraz z~kontrolera Kinect}; \\
		\node [above=of block3, below, yshift=-0.5cm](null1) {};  \\
		\node [decision, below=of block3] (inSync) {Czy dane zostały zsynchronizowane?}; \\
		\node [block, below=of inSync, xshift=3.5cm] (block4) {Synchronizacja czasowa sygnałów z~czujników inercyjnych oraz z~kontrolera Kinect}; \\
		\node [block, below=of inSync, xshift=-3.5cm] (block5) {Łączenie danych z~czujników inercyjnych oraz z~kontrolera Kinect}; \\
		\node [block, below=of inSync, yshift=-3.5cm] (block6) {Oszacowanie położenia stawów}; \\                    
	\end{scope}
	% connect all nodes defined above
	\begin{scope} [every path/.style=line]
		\path (block1)        --    (block2);
		\path (block2)      --   (block3);
		\path (block3)      --    (inSync);
		\path (inSync)  --++  (-3,0) node [near start, above] {Tak} -| (block5.north);
		\path (inSync)  --++  (3,0) node [near start, above] {Nie} -| (block4.north);
		\path (block5.south)   --++  (0,-1) node {}  -|    (block6.north);
		\path (block4.west) |- (block5.east);
		\path (block6)      --++  (9,0) node [near start] {} |-  (null1);
	\end{scope}			
\end{tikzpicture}