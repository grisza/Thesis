\documentclass[a4paper,12pt,openright,twoside,table,xcdraw]{mwbkfixed}
\usepackage[utf8]{inputenc}
\usepackage[OT4]{fontenc}
\usepackage[MeX]{polski}
\usepackage[polish]{babel}
\usepackage[style=ieee, citestyle=numeric-comp,backend=biber,sorting=nyt, defernumbers=true]{biblatex}
\usepackage{caption}
\usepackage{subcaption}
\usepackage{amsmath}
\usepackage{amsfonts}
\usepackage{numprint}
\usepackage{rotating}

\usepackage{blindtext}
\usepackage{bookmark}
\usepackage{hyperref}
\usepackage{ntheorem}
\usepackage{hyperref}
\usepackage{url}
\usepackage{graphicx}

\usepackage[usenames]{color}
\usepackage{enumerate}
\usepackage{pdfpages} 
\usepackage[notintoc, polish]{nomencl}
\usepackage{listings}
\usepackage{lscape} %do tekstu obróconego o~90 stopni
\usepackage{stmaryrd} %niektóre symbole matematyczne
\usepackage{indentfirst}
\usepackage{multirow}
\usepackage{booktabs}
\usepackage{float}
\usepackage{longtable} %łamanie tabel
\usepackage[toc,page]{appendix}
\usepackage{tikz}
\usepackage{enumitem}
\usepackage{textcomp}
\usepackage{multirow}
\usepackage{graphics}
\usepackage{gensymb}
\usepackage{epstopdf}
\usepackage{mathtools}
\usepackage{float}
\usepackage{sidecap}
\usepackage{textcomp}
\usepackage{pgfplots}
\usepackage{epstopdf}
\usepackage{listings}
\usepackage{array}
\usepackage{tabularx}
\usepackage{cases}
\usepackage{indentfirst}
\usepackage{MnSymbol}%
\usepackage{wasysym}%
\usepackage{footnote}


\addbibresource{main.bib}
\npdecimalsign{.}
\nprounddigits{2}


\DeclareGraphicsExtensions{.eps}
\usetikzlibrary{patterns}
\pgfplotsset{width=12cm,compat=1.14}
\usetikzlibrary{arrows,shapes,positioning,shadows,trees,shapes.geometric}
%%%%%%%%%%%%%%%%%%%%%%%%%%%%%

\definecolor{gray80}{gray}{0.8}
\definecolor{chartreuse(traditional)}{rgb}{0.87, 1.0, 0.0}
\definecolor{sepia}{rgb}{0.44, 0.26, 0.08}
\definecolor{turquoise}{rgb}{0.25 0.87 0.81}
\definecolor{mulberry}{rgb}{0.77 0.29 0.55}

\definecolor{codegreen}{rgb}{0,0.6,0}
\definecolor{codegray}{rgb}{0.5,0.5,0.5}
\definecolor{codepurple}{rgb}{0.58,0,0.82}
\definecolor{backcolour}{rgb}{0.95,0.95,0.92}

\begin{document}

\begin{savenotes}
	\begin{figure}[!htb]
		\centering
		\begin{tikzpicture}
	\begin{axis}[
			xlabel={Temperature [$\degree C$]},
			ylabel={Przyspieszenie ziemskie [g]},
			xmin=0, xmax=55,
			ymin=0.96, ymax=1.07,
			xtick={0,5,10,15,20,25,30,35,40,45,50,55},
			ytick={0.98,0.99,1,1.01,1.02,1.03,1.04,1.05},
			ymajorgrids=true,
			grid style=dashed,
		]
																																																																																																				
		\addplot+[
			only marks,
			scatter,
			color=blue,
			mark=square,
		]
		coordinates {
			(10,0.9842)(15,0.9895)(20,0.9957)(25,1.0002)(30,1.0077)(35,1.0122)(40, 1.0206)(45, 1.0232)(50, 1.0395)
		};
													
		\addplot [
			dashed,
			domain=8:51, 
			samples=90, 
			color=black,
		]
		{0.0000007*x^3 - 0.00005*x^2 + 0.0022*x + 0.9648};
		\legend{$y = 7*10^-7 x^3 - 5*10^-5 x^2 + 0.0022x + 0.9648$}
	\end{axis}
\end{tikzpicture}														
		\caption[Pomiar przyspieszenia ziemskiego w~przedziale temperatur 10-50$\degree C$]{Pomiar przyspieszenia ziemskiego w~przedziale temperatur 10-50$\degree C$ [źródło własne]}
		\label{fig:characteristics:imu:temp}
	\end{figure}
\end{savenotes}
					

\end{document}