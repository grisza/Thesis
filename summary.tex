\chapter{Podsumowanie i~wnioski }\label{chap:finalSummary}
%([Fixed] tak się powinien nazwyać ostatni rozdział i~najpierw co zostało zrbione a~ewentualnie po tym dopiero dalsze prace)


%([Fixed] WAŻNE w~podsumowaniu nie odniosłeś się w~ogóle do przyjętej tezty badawczej i~celów badań. One muszą być jasko i~precyzyjnie omóówione i~oczywiście trzeba pokazać, że przeproowadzone eksperymenty dowodzą tezy badawczej i~przy okazji została opracowana nowa metoda) -- dołożyłem kilka zwrotów wprost że coś sie zadziało zgodnie z~przyjęta tezą, tylko nie wiem czy to nie jest zbyt obcesowe.

Zaprezentowana w~pracy metoda zawiera nowe podejście do łączenia ze sobą danych z~czujników inercyjnych i~opisu modelu szkieletowego udostępnianego przez kotroler Kinect. W~wyniku hybrydowego łączenia danych o~orientacji kości układu szkieletowego, poprawiona została dokładność śledzenia ruchu stawów oraz możliwa była skuteczna aproksymacja informacji o~pozycji stawów szkieletu, których nie można było wyznaczyć bezpośrednio z~wykorzystanych urządzeń pomiarowych. Względem metod opisanych w~dostępnej i~znanej autorowi literaturze unikalnymi elementami zaproponowanej metody są:
\begin{itemize}
	\item \textbf{łączenie informacji o~obrotach kości},\\
	\item \textbf{kompleksowa kompensacja znanych i~mierzalnych błędów pomiarowych obu urządzeń}, \\
	\item \textbf{zróżnicowanie sposobu łączenia ze sobą danych w~zależności od jakości uzyskanych sygnałów, na który wpływa kontekst wykonywanego ruchu}.
\end{itemize}

%([Fixed] czy 12 czy 18 procent, bo mam wrażenie, że ostatnie wyniki się trochę różnią. Warto to uwspółnić z~publikacją) -- wszystko się zgadza, w~publikacji było że do 18% a~tuaj jest średnia która wychodzi koło 12. ale wersja z~publikacji jest bardziej adekwatna więc taką tutaj podaję.

Zgodnie z~przyjętą tezą i~celami badań, metoda opisana w~niniejszej pracy, łącząca ze sobą sygnały z~ogólnodostępnych, nieprofesjonalnych urządzeń pomiarowych jakimi są kontroler Kinect i~zintegrowane moduły inercyjne pokazuje, że odejście od łączenia informacji dotyczących położenia wybranych stawów na rzecz połączenia informacji o~tym jak względem osi układu współrzędnych obrócone sa związane z~nimi kości, zmniejsza błąd oszacowania położenia tychże stawów. Spowodowane jest to tym, że łączenie danych o~pozycjach śledzonych stawów, łączy także błędy związane z~określeniem długości poszczególnych kości modeli szkieletowych postaci wykorzystywanych przez oba te urządzenia. W~przypadku modułów inercyjnych umieszczonych na ciele, informacja o~ich orientacji przestrzennej (a co za tym idzie orientacji przestrzennej kości do której są przyczepione) jest niejako domyślną i~wymaga mniejszej ilości przetwarzania danych. Wyznaczenie położenia stawów powiązanych z~daną kością wymaga, oprócz wyznaczenia orientacji kości, dokładnego pomiaru jej długości tak, żeby wykorzystując przekształcenia geometryczne można było podać pozycję stawów. W~związku z~tym błąd takiego szacowania położenia stawu wynika z~niedokładności określania orientacji kości jak i~błędów pomiaru długości kości. W~przypadku kontrolera Kinect, model szkieletowy śledoznej postaci zawiera informacje o~długościach poszczególnych segmentów, jednak są to informacje szacowane na bierząco i~długość tej samej kości w~tym modelu może się różnić nawet o~kilka centymetrów pomiędzy następującymi po sobie pomiarami. Jednocześnie, jeśli w~trakcie śledzenia ruchu za pomocą Kinecta nie występuje okluzja, wówczas oszacowania orientacji poszczególnych kości są bardziej stabilne niż pozycji stawów. W~związku z~powyższym, łączenie szygnałów z~tych dwóch urządzeń pomiarowych: Kinecta i~modułów inercyjnych, wykorzystując informację o~obrotach, eliminuje istotne źródło błędu.\\

Uwzględnienie charakterystyk obu urządzeń pomiarowych oraz kompensacja mierzalnych błędów występujących w~ich pomiarach jest także elementem wyróżniającym proponowaną metodę na tle innych, opisanych w~literaturze i~pozwala istotnie poprawić dane uzyskane z~każdego z~urządzeń przed ich złączeniem. Istotne jest także to, że wiele z~tych charakterystyk jest związanych ściśle z~kontekstem wykonywanego przez użytkownika ruchu, np. charakterystyczna zmienność błędu szacowania odległości między kamerą a~użytkownikiem w~zależności gdzie się on znajduje, czy też to jak wpływa kąt obrotu sylwetki użytkownika względem płaszczyzny obserwacji Kinecta na stabilność jego pomiarów. Celem prowadzonych badań było opracowanie takiej metody, która uwzględniającej powyższe aspekty i~w~istotny sposób zmniejsza błąd oszacowania pozycji śledzonych stawów modelu szkieletowego człowieka. Zaproponowana metoda, przetestowana na przykładzie kończyn górnych, pozwoliła zmniejszyć błąd pozycjonowania stawów średnio do~18\% dla stawu łokciowego i~do 16\% dla stawu nadgarstkowego względem wyników uzyskiwanych za pomocą metody o~nominalnie największej deklarowanej w~artykułach dokładności, opartej o~łączenie danych o~pozycjach stawów oraz do~11\% mniejszy bład szacowania kąta zgięcia ręki w~łokciu ($\alpha$). Uzyskane wyniki charakteryzujące się mniejszym błędem szacowania pozycji stawów niż metoda opisana w~literaturze o~najmniejszym deklarowanym błędzie wyznaczania pozycji stawów, nie uwzględniająca jednak charakterystyk urządzeń i~kontekstu w~jakim wykonywany jest ruch, potwierdzają stawianą tezę, że metoda która bierze pod uwagę omawiane czynniki pozwala bardziej precyzyjnie śledzić wykonywany ruch.\\

Oczywiście opisywana metoda może być w~dalszym ciągu rozwijana i~modyfikowana. Celowe wydaje się być  sprawdzenie i~ewentualne dostosowanie omawianej metody wykorzystując kontroler Kinect w~wersji 2.0 (wersja dla konsoli Xbox One) działający w~oparciu o~inną technologię niż urządzenie użyte w~tej pracy. Odmienność tego urządzenia może sprawić, że część aktualnie wykonywanych kroków może być pominięta lub zastąpiona innymi. Innym możliwym kierunkiem badań jest poszerzenie spektrum ruchów charakterystycznych dla dyscyplin sportowych odznaczających się dużą dynamiką wykonywanych ruchów co z~kolei może mieć wpływ na dobór wartości wykorzystywanych jako kryterium sposobu łączenia sygnałów z~urządzeń pomiarowych, np. większa fragmentacja danych przy szybkim ruchu kończyny może mieć wpływ na wariancję prawidłowych pomiarów w~określonym przedziale czasu.\\
Duże znaczenie dla przeprowadzonych badań miały warunki, zwłaszcza oświetleniowe, jakie panowały w~laboratorium. Zdarzyło się kilkukrotnie, że ćwiczenia musiały zostać przerwane, bądź to ze względu na unoszenie się zanieczyszczenia niewiadomego pochodzenia w~powietrzu, które odbijało światło w~sposób zakłócający działanie Kinecta, bądź też ze względu na zakłócenia jakim poddana była transmisja danych za pomocą Bluetooth. Objawiało się to tym, że do komputera, który odbierał i~rejestrował wszystkie dane, pakiety z~pomiarami IMU trafiały niekompletne lub uszkodzone. Elementami które wymagać mogą więc dalszego rozwoju jest np. poprawa formatu danych na taki, który byłby w~stanie je odtworzyć w~przypadku niewielkiego uszkodzenia w~trakcie transmisji.\\
%([Fixed] Ten akapit mi się podoba bo definiuje problemy podczas badań i~daleko idące usprawnienia mocno wykraczające poza główny aspekt pracy).\\ :)
Ostatnim z~rozważanych przeze mnie kierunków dalszych badań jest modyfikacja urządzenia agregującego sygnały z~czujników inercyjnych. Modyfikacja ta polegałaby na zamiany obecnie wykorzystywanego modelu opartego o~jeden moduł centralny z~modłączonymi do niego modułami inercyjnymi i~zapewniającego komunikację z~komputerem PC (rys. \ref{fig:device:circuitDiagram}) na rzecz modelu opartego o~wiele niezależnych, autonomicznych modułów inercyjnych, z~których każdy komunikuje się niezależnie z~komputerem PC, na którym odbywa się łączenie wszystkich zagreowanych pomiarów. Głównym powodem rozważanej modyfikacji jest przede wszystkim wygoda użytkowania systemu poprzez wyeliminowanie przewodów łączących moduły inercyjne z~modułem centralnym. Z~drugiej jednak strony, taka zmiana spowodowałaby potencjalne problemy synchronizacji sygnałów pomiędzy nawet kilkunastoma urządzeniami pomiarowymi co mogłoby wprowadzić istotne błędy w~śledzeniu ruchu postaci.
%([Fixed] czy agregujący?) 


%([Fixed] zamiast osobnego rozdziału w~tym miejscy wpleść konkretne propozycje dalszych badań)\\




%Metoda opisana w~niniejszej pracy pokazuje, że możliwe jest wykorzystanie ogólnodostępnych, nieprofesjonalnych i~niedrogich urządzeń do śledzenia ruchu kończyn i~po połączeniu ich sygnałów uzyskać wyniki poprawiające i~bardziej kompletne niż te uzyskiwane przez każde z~urządzeń z~osobna.
%([Fixed] to trochę mało odkrywcze, bo ogólnie wiadomo, że data fuzion poprawia wyniki w~porównaiu z~każdym urządzenime z~osobna). 
%Przeprowadzone, z~użyciem autorskiej metody, eksperymenty dowodzą również, że połączenie danych o~obrotach kości daje poprawę szacowania położenia stawów w~porównaniu z~metodami opartymi o~bezpośrednie śledzenie pozycji stawów. Dzieje się to dlatego, że w~trakcie łączenia informacji o~obrocie kości
%([Fixed] czego?)
% nie dodajemy potencjalnego błędu wynikającego z~niedokładności pomiaru modelu długości kości, który ma bezpośredni wpływ na wyznaczenie położenia stawów.
%([Fixed] mało zrozumiałe czego to dotyczy). 
%Dodatkowo eliminujemy w~ten sposób ([Fixed] w~jaki sposób? O czym jest mowa?) problem, który widoczny jest w~przypadku samego Kinecta tj. zmiennej długości każdej z~kości w~zależności od aktualnej klatki ([Fixed] a~w~jaki sposób eliminujemy to, że Kinect zmiania szacowania długości kości. Chyba trochę inaczej to trzeba napisać). Oczywiście metoda ta może być w~dalszym ciągu rozwijana i~modyfikowana czego zidentyfikowane pomysły zostały przedstawione w~rozdziale \ref{chap:durther}.([Fixed] zamiast osobnego rozdziału w~tym miejscy wpleść konkretne propozycje dalszych badań)\\

%Przyjęty sposób realizacji łączenia danych na potrzeby śledzenia ruchu stawów szkieletu człowieka pokazuje, że choć dotychczas nie opisywany w~literaturze, jest kierunkiem który warto rozwijać i~ulepszać. ([Fixed] to zdanie jest chyba nie potrzebne)


%\paragraph{Dalsze prace}\label{chap:durther}
%([Fixed] ja bym rozdziału nie nazywał dalsze prace tylko wszystko jako podsumowanie, szczególnie że oba rozdziały z~części 5 są krótkie. Poza tym nie politycznie jest zaczynać podsumowanie od tego co można jeszcze zrobić. Trzeba do bólu podkreślać co się zrobiło i~jakie to może miec znaczenie dla ludzkośći)
%Wyniki uzyskane za pomocą metody przedstawionej w~niniejszej pracy wykazują przewagę opracowanego rozwiązania nad innymi metodami opisanymi współcześnie w~literaturze. ([Fixed] Nie można rozpoczynać podsumowania od tego co będzie zrobione. Trzeba podkreślać to co zostało zrobione) W~perspektywie dalszych badań celowe jest sprawdzenie i~ewentualne dostosowanie metody do danych pochodzących z~drugiej wersji Kinecta działającego w~oparciu o~inną technologię niż urządzenie użyte w~tej pracy. Odmienność tego urządzenia może sprawić, że część aktualnie wykonywanych kroków może być pominięta lub zastąpiona innymi. \\

%Kolejnym kierunkiem badań jaki można podjąć to dostosowanie algorytmu do śledzenia innych części ciała ([Fixed] to jest strzałw stopę, bo metoda musi być uniwerslana dla każdej kończyny inaczej nie jest metodą). Ze względu na czas prowadzonych badań te skupiły się na dostosowaniu metody do ruchu ręki ([Fixed] nie można tak pisać). Mając świadomość specyfiki ruchu każdej z~kończyn: szybkości wykonywanego ruchu, wielkości samej kończyny czy zakresu w~jakim się ona porusza, metoda może wymagać przedefiniowania warunków, dla których zebrane dane uznajemy za wiarygodne.([Fixed] takiego czegoś bym nie pisał. Bezpiecznie jest wspomnieć o~szybkości ruchu, w~którym dane będą znacznie bardziej pofragmentowane, ale ich wiarygodność nie powinna się zmienić)\\

%Z punktu widzenia prowadzonych ([Fixed] kto to jest prowadzący?) w~trakcie działania metody obliczeń zasadnym wydaje się przeprowadzenie dodatkowych badań związanych z~bezpośrednim łączeniem ze sobą kwaternionów w~taki sposób aby można było nadać różne wartości wag dla każdej z~osi obrotu ([Fixed] Wpisując tak oczywiste i~relatywnie krótkotrwałe rozszerzenia dotychczasoych osiągnąć ktoś może łątwo zapytać a~dlaczego to nie zostało zrobione, skoro jest oczywiste. Swoją drogą warto to może kiedyś zrobić.). Zastosowana w~niniejszej pracy transformacja do postaci kątów Eulera, choć czytelna dla programisty, z~punktu widzenia prowadzonych obliczeń może być nie efektywna. Hipotetycznie jej użycie może doprowadzić do problemu powstania niejednoznaczności obrotu (\emph{ang. gimbal lock}) jednak w~trakcie eksperymentów do takiej sytuacji nie doszło ([Fixed] to znowu kopanie się po kostkach).\\


%Wreszcie zasadnym wydaje się ([Fixed] zasadne, czyli można przeczytać jako niezbędne. MOże jakoś mniej zobowiązująco, np. Kolejnym etapem badań będzie...) zweryfikowanie działania metody stosując inne, alternatywne kotrolery jak np.: ([Fixed]trzeba konkretnie podać)(z~innymi urządzeniami tego samego rodzaju co te zaprezentowane w~niniejszej pracy). Szczególnie interesujący wydaje się być sposób działania ([Fixed] czego?)z~wykorzystaniem analogowych IMU ponieważ mogą one wprowadzić konieczność dodatkowych kroków filtrujących te dane przed dalszymi obliczeniami.([Fixed] ostatnie zdanie brzmiało, że jak zastosujemy analogowe IMU to będzie więcej roboty - to chyba nie tak miało brzmieć)\\

