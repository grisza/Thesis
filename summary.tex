\chapter{Podsumowanie i~wnioski }\label{chap:finalSummary}

Zaprezentowana w~pracy metoda zawiera nowe podejście do łączenia ze sobą danych z~czujników inercyjnych i~opisu modelu szkieletowego udostępnianego przez kotroler Kinect. W~wyniku hybrydowego łączenia danych o~orientacji kości układu szkieletowego, poprawiona została dokładność śledzenia ruchu stawów oraz możliwa była skuteczna aproksymacja informacji o~pozycji stawów szkieletu, których nie można było wyznaczyć bezpośrednio z~wykorzystanych urządzeń pomiarowych. Względem metod opisanych w~dostępnej i~znanej autorowi literaturze unikalnymi elementami zaproponowanej metody są:
\begin{itemize}
	\item \textbf{łączenie informacji na etapie szacowania orientacji kości i~szacowanie pozycji stawów na podstawie wypadkowej wartości ich orientacji},\\
	\item \textbf{kompleksowa kompensacja znanych i~mierzalnych błędów pomiarowych obu urządzeń}, \\
	\item \textbf{zróżnicowanie sposobu łączenia ze sobą danych w~zależności od jakości uzyskanych sygnałów, na który wpływa kontekst wykonywanego ruchu}.
\end{itemize}


Zgodnie z~przyjętą tezą i~celami badań, metoda opisana w~niniejszej pracy, łącząca ze sobą sygnały z~ogólnodostępnych, nieprofesjonalnych urządzeń pomiarowych, jakimi są kontroler Kinect i~zintegrowane moduły inercyjne pokazuje, że odejście od łączenia informacji dotyczących położenia wybranych stawów na rzecz połączenia informacji o~tym jak względem osi układu współrzędnych obrócone sa związane z~nimi kości, zmniejsza błąd oszacowania położenia tychże stawów. Spowodowane jest to tym, że łączenie danych o~pozycjach śledzonych stawów, uwypukla błędy związane z~określeniem długości poszczególnych kości modeli szkieletowych postaci wykorzystywanych przez oba te urządzenia. W~przypadku modułów inercyjnych umieszczonych na ciele, informacja o~ich orientacji przestrzennej (a co za tym idzie orientacji przestrzennej kości, do której są przyczepione) jest niejako domyślną i~wymaga mniej złożonego przetwarzania danych. Wyznaczenie położenia stawów powiązanych z~daną kością wymaga, oprócz wyznaczenia orientacji kości, dokładnego pomiaru jej długości tak, żeby wykorzystując przekształcenia geometryczne można było podać pozycję stawów. W~związku z~tym błąd takiego szacowania położenia stawu wynika z~niedokładności określania orientacji kości jak i~błędów pomiaru długości kości. W~przypadku kontrolera Kinect, model szkieletowy śledzonej postaci zawiera informacje o~długościach poszczególnych segmentów, jednak są to informacje szacowane na bieżąco i~długość tej samej kości w~tym modelu może się różnić nawet o~kilka centymetrów pomiędzy następującymi po sobie pomiarami. Jednocześnie, jeśli w~trakcie śledzenia ruchu, za pomocą Kinecta nie występuje okluzja, wówczas oszacowania orientacji poszczególnych kości są bardziej stabilne niż pozycji stawów. W~związku z~powyższym, łączenie sygnałów z~tych dwóch urządzeń pomiarowych: Kinecta i~modułów inercyjnych, wykorzystując informację o~obrotach, eliminuje istotne źródło błędu.\\

Uwzględnienie charakterystyk obu urządzeń pomiarowych oraz kompensacja mierzalnych błędów, występujących w~ich pomiarach, jest także elementem wyróżniającym proponowaną metodę na tle innych, opisanych w~literaturze i~pozwala istotnie poprawić dane uzyskane z~każdego z~urządzeń przed ich złączeniem. Istotne jest także to, że wiele z~tych charakterystyk jest związanych ściśle z~kontekstem wykonywanego przez użytkownika ruchu, przykładowo charakterystyczna zmienność błędu szacowania odległości między kamerą a~użytkownikiem, w~zależności gdzie się on znajduje, czy też to jak wpływa kąt obrotu sylwetki użytkownika względem płaszczyzny obserwacji Kinecta na stabilność jego pomiarów. Celem prowadzonych badań było opracowanie takiej metody, która uwzględniającej powyższe aspekty i~w~istotny sposób zmniejsza błąd oszacowania pozycji śledzonych stawów modelu szkieletowego człowieka. Zaproponowana metoda, przetestowana na przykładzie kończyn górnych, pozwoliła zmniejszyć błąd pozycjonowania stawów średnio do~18\% dla stawu łokciowego i~do 16\% dla stawu nadgarstkowego względem wyników uzyskiwanych za pomocą metody o~nominalnie największej deklarowanej w~artykułach dokładności, oraz do~11\% mniejszy błąd szacowania kąta zgięcia ręki w~łokciu ($\beta$). Uzyskane wyniki charakteryzujące się mniejszym błędem szacowania pozycji stawów niż metody opisane w~literaturze, potwierdzają stawianą tezę, że metoda która bierze pod uwagę omawiane czynniki pozwala bardziej precyzyjnie śledzić wykonywany ruch.\\

Oczywiście opisywana metoda może być w~dalszym ciągu rozwijana i~modyfikowana. Celowe wydaje się sprawdzenie i~ewentualne dostosowanie omawianej metody do obsługi kontrolera Kinect w~wersji 2.0 (wersja dla konsoli Xbox One) działającego w~oparciu o~inną technologię, niż urządzenie użyte w~niniejszej pracy. Odmienność tego urządzenia może sprawić, że część aktualnie wykonywanych kroków może być pominięta lub zastąpiona innymi. Innym możliwym kierunkiem badań jest poszerzenie spektrum ruchów, charakterystycznych dla dyscyplin sportowych odznaczających się dużą dynamiką wykonywanych ruchów, co z~kolei może mieć wpływ na dobór wartości wykorzystywanych jako kryterium sposobu łączenia sygnałów z~urządzeń pomiarowych, przykładowo większa fragmentacja danych przy szybkim ruchu kończyny może mieć wpływ na wariancję prawidłowych pomiarów w~określonym przedziale czasu.\\
Duże znaczenie dla przeprowadzonych badań miały warunki, zwłaszcza oświetleniowe, jakie panowały w~laboratorium. Zdarzyło się kilkukrotnie, że ćwiczenia musiały zostać przerwane, bądź to ze względu na unoszenie się zanieczyszczenia niewiadomego pochodzenia w~powietrzu, które odbijało światło w~sposób zakłócający działanie Kinecta, bądź też ze względu na zakłócenia jakim poddana była transmisja danych za pomocą technologii  komunikacji bezprzewodowej Bluetooth. Objawiało się to tym, że do komputera, który odbierał i~rejestrował wszystkie dane, pakiety z~pomiarami IMU trafiały niekompletne lub uszkodzone. Elementami, które wymagać mogą więc dalszego rozwoju jest na przykład poprawa formatu danych na taki, który byłby w~stanie je odtworzyć w~przypadku niewielkiego uszkodzenia w~trakcie transmisji.\\
%([Fixed] Ten akapit mi się podoba bo definiuje problemy podczas badań i~daleko idące usprawnienia mocno wykraczające poza główny aspekt pracy).\\ :)
Ostatnim z~rozważanych przeze mnie kierunków dalszych badań jest modyfikacja urządzenia agregującego sygnały z~czujników inercyjnych. Modyfikacja ta polegałaby na zamiany obecnie wykorzystywanego modelu opartego o~jeden moduł centralny z~podłączonymi do niego modułami inercyjnymi i~zapewniającego komunikację z~komputerem PC (rys. \ref{fig:device:circuitDiagram}) na rzecz modelu opartego o~wiele niezależnych, autonomicznych modułów inercyjnych, z~których każdy komunikuje się niezależnie z~komputerem PC, na którym odbywa się łączenie wszystkich zagreowanych pomiarów. Głównym powodem rozważanej modyfikacji jest przede wszystkim wygoda użytkowania systemu poprzez wyeliminowanie przewodów łączących moduły inercyjne z~modułem centralnym. Z~drugiej jednak strony, taka zmiana spowodowałaby potencjalne problemy synchronizacji sygnałów pomiędzy nawet kilkunastoma urządzeniami pomiarowymi co mogłoby wprowadzić istotne błędy w~śledzeniu ruchu postaci.